\usepackage{multicol}
\usepackage{booktabs}

\usepackage{xcolor}
\usepackage{tikz}
\usepackage[linewidth=1pt,framemethod=TikZ,fontcolor=white]{mdframed}

\usepackage{braket}
\usepackage{bm}
%bold symbols
\newcommand{\mb}[1]{\boldsymbol{#1}}
\newcommand{\br}{\mb{r}}
\newcommand{\bx}{\mb{x}}
\newcommand{\ud}{\text{d}}

\usepackage{tcolorbox}

\newtcbox{\mybox}{nobeforeafter,colframe=greenlink,colback=greenlink,boxrule=0.8pt,%arc=10pt,
  boxsep=0pt,left=10pt,right=10pt,top=6pt,bottom=6pt,tcbox raise base}

\usepackage{chemfig}
\usepackage[version=4]{mhchem}

%\usepackage[backend=biber,style=chem-angew]{biblatex}
%\addbibresource{library.bib}
%\setbeamertemplate{bibliography item}[text]

\usepackage{tabularx}
\newcolumntype{C}{>{\centering\arraybackslash}X}
\newcolumntype{L}{>{\raggedright\arraybackslash}X}
\newcolumntype{R}{>{\raggedleft\arraybackslash}X}

\newcommand{\ans}[1]{{\color{llgreen}\textbf{#1}}}

\newcommand{\lenitem}[2][.7\linewidth]{\parbox[t]{#1}{\strut #2\strut}}

\mdfdefinestyle{myFigureBoxStyle}{backgroundcolor=codebg, roundcorner=5pt, linewidth=0pt}%

% minted and syntax highlighting
\usepackage{minted}
\usemintedstyle{monokai}
\setminted{fontsize=\footnotesize,escapeinside=||,bgcolor=codebg}
