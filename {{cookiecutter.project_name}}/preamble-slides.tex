\usetheme[titleformat=regular,%
          progressbar=foot,%
          block=transparent,%
          numbering=counter,%
          background=dark]{metropolis}

\usepackage{multicol}
\usepackage{booktabs}
% minted and syntax highlighting
%\usepackage{minted}
%\usemintedstyle{monokai}
%\setminted{fontsize=\footnotesize,escapeinside=||,bgcolor=codebg}
%\newcommand{\pyp}{>\null>\null>}
\usepackage{xcolor}
\usepackage{tikz}
\usepackage[linewidth=1pt,framemethod=TikZ,fontcolor=white]{mdframed}

\definecolor{Nero}{HTML}{222222}
\definecolor{rred}{HTML}{FF2c2d}
\definecolor{rblue}{HTML}{1B91FF}
\definecolor{dblue}{HTML}{23373B}
\definecolor{rgreen}{HTML}{2E623C}
\definecolor{dgreen}{HTML}{2A4832}
\definecolor{lgreen}{HTML}{239541}
\definecolor{llgreen}{HTML}{14B03D}
\definecolor{fijigreen}{HTML}{4A620C}
\definecolor{dorange}{HTML}{F7921D}
\definecolor{processblue}{HTML}{00B0F0}

\definecolor{crimson}{HTML}{F92672}
\definecolor{fgreen}{HTML}{14B03D}
\definecolor{redbg}{HTML}{D46B40}
\definecolor{codebg}{HTML}{3F3F3F}
\definecolor{greenbg}{HTML}{14B03D}
\definecolor{bluebg}{HTML}{3873B3}
\definecolor{greenlink}{HTML}{AAD33B}
\definecolor{dimgrey}{HTML}{9B9B9B}

% change the default color palette
\setbeamercolor{palette primary}{fg=white, bg=Nero}
\setbeamercolor{normal text}{fg=white, bg=Nero}
\setbeamercolor{title}{fg=crimson, bg=Nero}
\setbeamercolor{footnote}{fg=dimgrey}
\setbeamercolor{footnote mark}{fg=dimgrey}
\setbeamercolor{footline}{fg=dimgrey}

% change the color of the progress bar
\setbeamercolor{progress bar}{fg=crimson, bg=Nero}

% ascending fontsizes:
% \tiny 
% \scriptsize
% \footnotesize
% \small
% \normalsize
% \large
% \Large
% \LARGE
% \huge
% \Huge



% adjust fonts
% good alternative to roboto is lato
\setsansfont[BoldFont={Roboto Regular}, ItalicFont={Roboto Italic}]{Roboto Light}
\newfontfamily\Titles[BoldFont={Yanone Kaffeesatz Bold}]{Yanone Kaffeesatz Regular}
\setbeamerfont{title}{family=\Titles, size=\Huge, series=\normalfont}
\setbeamerfont{subtitle}{family=\Titles, size=\normalsize, series=\normalfont}
\setbeamerfont{frametitle}{family=\Titles, size=\LARGE, series=\normalfont}
\setbeamerfont{section title}{family=\Titles, size=\Large}
\setbeamerfont{standout}{family=\Titles, size=\LARGE, series=\bfseries}

% customize blocks
\setbeamerfont{block title}{family=\Titles, size=\large}
\setbeamercolor{block title}{fg=rblue, bg=Nero}

% set all the frametitle to be centered
\setbeamertemplate{frametitle}[default][center]

\usepackage{braket}
\usepackage{bm}
%bold symbols
\newcommand{\mb}[1]{\boldsymbol{#1}}
\newcommand{\br}{\mb{r}}
\newcommand{\bx}{\mb{x}}
\newcommand{\ud}{\text{d}}

\usepackage{tcolorbox}

\newtcbox{\mybox}{nobeforeafter,colframe=greenlink,colback=greenlink,boxrule=0.8pt,%arc=10pt,
  boxsep=0pt,left=10pt,right=10pt,top=6pt,bottom=6pt,tcbox raise base}

%\usepackage{hyperref}

%\makeatletter
%\Hy@AtBeginDocument{%
%  \def\@pdfborder{0 0 1}% Overrides border definition set with colorlinks=true
%  \def\@pdfborderstyle{/S/U/W 1}% Overrides border style set with colorlinks=true
%                                % Hyperlink border style will be underline of width 1pt
%}
%\makeatother


%\usepackage{tikz}
%\usetikzlibrary{mindmap,graphdrawing,graphs,trees,shadows}
%\usegdlibrary{layered,trees,circular,force}

\usepackage{chemfig}
\usepackage[version=4]{mhchem}

%\setbeamercolor{footnote mark}{fg=red}
%\setbeameroption{show only notes}

%\usepackage[backend=biber,style=chem-angew]{biblatex}
%\addbibresource{library.bib}
%\setbeamertemplate{bibliography item}[text]

\usepackage{tabularx}
\newcolumntype{C}{>{\centering\arraybackslash}X}
\newcolumntype{L}{>{\raggedright\arraybackslash}X}
\newcolumntype{R}{>{\raggedleft\arraybackslash}X}

\newcommand{\ans}[1]{{\color{llgreen}\textbf{#1}}}

\newcommand{\lenitem}[2][.7\linewidth]{\parbox[t]{#1}{\strut #2\strut}}

\mdfdefinestyle{myFigureBoxStyle}{backgroundcolor=codebg, roundcorner=5pt, linewidth=0pt}%
