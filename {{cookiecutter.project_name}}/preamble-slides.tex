\usetheme[titleformat=regular,%
          progressbar=foot,%
          block=transparent,%
          numbering=counter,%
          background=dark]{metropolis}

\usepackage{multicol}
\usepackage{booktabs}

\usepackage{xcolor}
\usepackage{tikz}
\usepackage[linewidth=1pt,framemethod=TikZ,fontcolor=white]{mdframed}

\definecolor{Nero}{HTML}{222222}
\definecolor{crimson}{HTML}{F92672}
\definecolor{rred}{HTML}{FF2c2d}
\definecolor{redbg}{HTML}{D46B40}
\definecolor{rblue}{HTML}{1B91FF}
\definecolor{fgreen}{HTML}{14B03D}
\definecolor{lgreen}{HTML}{239541}
\definecolor{fijigreen}{HTML}{4A620C}
\definecolor{dorange}{HTML}{F7921D}
\definecolor{processblue}{HTML}{00B0F0}
\definecolor{codebg}{HTML}{3F3F3F}
\definecolor{bluebg}{HTML}{3873B3}
\definecolor{greenlink}{HTML}{AAD33B}
\definecolor{dimgrey}{HTML}{9B9B9B}

% change the default color palette
\setbeamercolor{palette primary}{fg=white, bg=Nero}
\setbeamercolor{normal text}{fg=white, bg=Nero}
\setbeamercolor{title}{fg=crimson, bg=Nero}
\setbeamercolor{footnote}{fg=dimgrey}
\setbeamercolor{footnote mark}{fg=dimgrey}
\setbeamercolor{footline}{fg=dimgrey}

% change the color of the progress bar
\setbeamercolor{progress bar}{fg=crimson, bg=Nero}

% adjust fonts
% good alternative to roboto is lato
\setsansfont[BoldFont={Roboto Regular}, ItalicFont={Roboto Italic}]{Roboto Light}
\newfontfamily\Titles[BoldFont={Yanone Kaffeesatz Bold}]{Yanone Kaffeesatz Regular}
\setbeamerfont{title}{family=\Titles, size=\Huge, series=\normalfont}
\setbeamerfont{subtitle}{family=\Titles, size=\normalsize, series=\normalfont}
\setbeamerfont{frametitle}{family=\Titles, size=\LARGE, series=\normalfont}
\setbeamerfont{section title}{family=\Titles, size=\Large}
\setbeamerfont{standout}{family=\Titles, size=\LARGE, series=\bfseries}

% alerted text
\setbeamerfont{alerted text}{family=\Titles}
\setbeamercolor{alerted text}{fg=crimson, bg=Nero}

% customize blocks
\setbeamerfont{block title}{family=\Titles, size=\large}
\setbeamercolor{block title}{fg=rblue, bg=Nero}

% set all the frametitle to be centered
\setbeamertemplate{frametitle}[default][center]

\usepackage{braket}
\usepackage{bm}
%bold symbols
\newcommand{\mb}[1]{\boldsymbol{#1}}
\newcommand{\br}{\mb{r}}
\newcommand{\bx}{\mb{x}}
\newcommand{\ud}{\text{d}}

\usepackage{tcolorbox}

\newtcbox{\mybox}{nobeforeafter,colframe=greenlink,colback=greenlink,boxrule=0.8pt,%arc=10pt,
  boxsep=0pt,left=10pt,right=10pt,top=6pt,bottom=6pt,tcbox raise base}

\usepackage{chemfig}
\usepackage[version=4]{mhchem}

%\usepackage[backend=biber,style=chem-angew]{biblatex}
%\addbibresource{library.bib}
%\setbeamertemplate{bibliography item}[text]

\usepackage{tabularx}
\newcolumntype{C}{>{\centering\arraybackslash}X}
\newcolumntype{L}{>{\raggedright\arraybackslash}X}
\newcolumntype{R}{>{\raggedleft\arraybackslash}X}

\newcommand{\ans}[1]{{\color{llgreen}\textbf{#1}}}

\newcommand{\lenitem}[2][.7\linewidth]{\parbox[t]{#1}{\strut #2\strut}}

\mdfdefinestyle{myFigureBoxStyle}{backgroundcolor=codebg, roundcorner=5pt, linewidth=0pt}%

\usepackage{listings}
%\usepackage{fontspec}
%\setmonofont{Consolas}

\definecolor{background}{RGB}{39, 40, 34}
\definecolor{string}{RGB}{230, 219, 116}
\definecolor{comment}{RGB}{117, 113, 94}
\definecolor{normal}{RGB}{248, 248, 242}
\definecolor{identifier}{RGB}{166, 226, 46}

\lstset{
  language=python,                          % choose the language of the code
  numbers=left,                         % where to put the line-numbers
  stepnumber=1,                         % the step between two line-numbers.        
  numbersep=5pt,                        % how far the line-numbers are from the code
  numberstyle=\tiny\color{black}\ttfamily,
  backgroundcolor=\color{background},       % choose the background color. You must add \usepackage{color}
  showspaces=false,                     % show spaces adding particular underscores
  showstringspaces=false,               % underline spaces within strings
  showtabs=false,                       % show tabs within strings adding particular underscores
  tabsize=4,                            % sets default tabsize to 2 spaces
  captionpos=b,                         % sets the caption-position to bottom
  breaklines=true,                      % sets automatic line breaking
  breakatwhitespace=true,               % sets if automatic breaks should only happen at whitespace
  title=\lstname,                       % show the filename of files included with \lstinputlisting;
  basicstyle=\color{normal}\ttfamily,                   % sets font style for the code
  keywordstyle=\color{magenta}\ttfamily,    % sets color for keywords
  stringstyle=\color{string}\ttfamily,      % sets color for strings
  commentstyle=\color{comment}\ttfamily,    % sets color for comments
  emph={format_string, eff_ana_bf, permute, eff_ana_btr},
  emphstyle=\color{identifier}\ttfamily
}